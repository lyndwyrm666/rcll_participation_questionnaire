\documentclass[a4paper,10pt,BCOR10mm,oneside,headsepline]{scrartcl}
\usepackage[ngerman]{babel}
\usepackage[utf8]{inputenc}
\usepackage{wasysym}% provides \ocircle and \Box
\usepackage{enumitem}% easy control of topsep and leftmargin for lists
\usepackage{color}% used for background color
\usepackage{forloop}% used for \Qrating and \Qlines
\usepackage{ifthen}% used for \Qitem and \QItem
\usepackage{typearea}
\areaset{17cm}{26cm}
\setlength{\topmargin}{-1cm}
\usepackage{scrpage2}
\pagestyle{scrheadings}
\ihead{RCLL Participation Questionnaire}
\ohead{\pagemark}
\chead{}
\cfoot{}

%%%%%%%%%%%%%%%%%%%%%%%%%%%%%%%%%%%%%%%%%%%%%%%%%%%%%%%%%%%%
%% Beginning of questionnaire command definitions %%
%%%%%%%%%%%%%%%%%%%%%%%%%%%%%%%%%%%%%%%%%%%%%%%%%%%%%%%%%%%%
%%
%% 2010, 2012 by Sven Hartenstein
%% mail@svenhartenstein.de
%% http://www.svenhartenstein.de
%%
%% Please be warned that this is NOT a full-featured framework for
%% creating (all sorts of) questionnaires. Rather, it is a small
%% collection of LaTeX commands that I found useful when creating a
%% questionnaire. Feel free to copy and adjust any parts you like.
%% Most probably, you will want to change the commands, so that they
%% fit your taste.
%%
%% Also note that I am not a LaTeX expert! Things can very likely be
%% done much more elegant than I was able to. If you have suggestions
%% about what can be improved please send me an email. I intend to
%% add good tipps to my website and to name contributers of course.
%%
%% 10/2012: Thanks to karathan for the suggestion to put \noindent
%% before \rule!

%% \Qq = Questionaire question. Oh, this is just too simple. It helps
%% making it easy to globally change the appearance of questions.
\newcommand{\Qq}[1]{\textbf{#1}}

%% \QO = Circle or box to be ticked. Used both by direct call and by
%% \Qrating and \Qlist.
\newcommand{\QO}{$\Box$}% or: $\ocircle$

%% \Qrating = Automatically create a rating scale with NUM steps, like
%% this: 0--0--0--0--0.
\newcounter{qr}
\newcommand{\Qrating}[1]{\QO\forloop{qr}{1}{\value{qr} < #1}{---\QO}}

%% \Qline = Again, this is very simple. It helps setting the line
%% thickness globally. Used both by direct call and by \Qlines.
\newcommand{\Qline}[1]{\noindent\rule{#1}{0.6pt}}

%% \Qlines = Insert NUM lines with width=\linewith. You can change the
%% \vskip value to adjust the spacing.
\newcounter{ql}
\newcommand{\Qlines}[1]{\forloop{ql}{0}{\value{ql}<#1}{\vskip0em\Qline{\linewidth}}}

%% \Qlist = This is an environment very similar to itemize but with
%% \QO in front of each list item. Useful for classical multiple
%% choice. Change leftmargin and topsep accourding to your taste.
\newenvironment{Qlist}{%
\renewcommand{\labelitemi}{\QO}
\begin{itemize}[leftmargin=1.5em,topsep=-.5em]
}{%
\end{itemize}
}

%% \Qtab = A "tabulator simulation". The first argument is the
%% distance from the left margin. The second argument is content which
%% is indented within the current row.
\newlength{\qt}
\newcommand{\Qtab}[2]{
\setlength{\qt}{\linewidth}
\addtolength{\qt}{-#1}
\hfill\parbox[t]{\qt}{\raggedright #2}
}

%% \Qitem = Item with automatic numbering. The first optional argument
%% can be used to create sub-items like 2a, 2b, 2c, ... The item
%% number is increased if the first argument is omitted or equals 'a'.
%% You will have to adjust this if you prefer a different numbering
%% scheme. Adjust topsep and leftmargin as needed.
\newcounter{itemnummer}
\newcommand{\Qitem}[2][]{% #1 optional, #2 notwendig
\ifthenelse{\equal{#1}{}}{\stepcounter{itemnummer}}{}
\ifthenelse{\equal{#1}{a}}{\stepcounter{itemnummer}}{}
\begin{enumerate}[topsep=2pt,leftmargin=2.8em]
\item[\textbf{\arabic{itemnummer}#1.}] #2
\end{enumerate}
}

%% \QItem = Like \Qitem but with alternating background color. This
%% might be error prone as I hard-coded some lengths (-5.25pt and
%%% -3pt)! I do not yet understand why I need them.
\definecolor{bgodd}{rgb}{0.8,0.8,0.8}
\definecolor{bgeven}{rgb}{0.9,0.9,0.9}
\newcounter{itemoddeven}
\newlength{\gb}
\newcommand{\QItem}[2][]{% #1 optional, #2 notwendig
\setlength{\gb}{\linewidth}
\addtolength{\gb}{-5.25pt}
\ifthenelse{\equal{\value{itemoddeven}}{0}}{%
\noindent\colorbox{bgeven}{\hskip-3pt\begin{minipage}{\gb}\Qitem[#1]{#2}\end{minipage}}%
\stepcounter{itemoddeven}%
}{%
\noindent\colorbox{bgodd}{\hskip-3pt\begin{minipage}{\gb}\Qitem[#1]{#2}\end{minipage}}%
\setcounter{itemoddeven}{0}%
}
}

%%%%%%%%%%%%%%%%%%%%%%%%%%%%%%%%%%%%%%%%%%%%%%%%%%%%%%%%%%%%
%% End of questionnaire command definitions %%
%%%%%%%%%%%%%%%%%%%%%%%%%%%%%%%%%%%%%%%%%%%%%%%%%%%%%%%%%%%%

\begin{document}

\begin{center}
\textbf{\Large Questionnare: Participation in the RoboCup Logistics League}
\end{center}\vskip1em

\noindent In this questionnaire we want to evaluate how to make our league more accessible. Not every section will be relevant to you, depending on your experience with RoboCup. If there is not enough room for your answers, you can use the back page.

\section*{General Background}

\Qitem{ \Qq{What have you heard of Industry 4.0 before?} \hfill  \Qline{7cm} \newline \vskip 1pt \Qline{16cm}}

\Qitem{ \Qq{Are you interested in the concept of Industry 4.0?} \hfill \QO{} Yes\hskip1.25cm \QO{} No}

\Qitem[a]{ \Qq{Have you heard of the RCLL before?} \hfill \QO{} Yes\hskip1.25cm \QO{} No}

\Qitem[b]{ \Qq{Have you seen a RCLL game before?} \hfill \QO{} Yes\hskip1.25cm \QO{} No}

\section*{If you are new to RoboCup}

\Qitem{ \Qq{Do you think RoboCup is an interesting concept?} \hfill \QO{} Yes\hskip1.25cm \QO{} No}

\Qitem{ \Qq{What RoboCup Leagues would you most likely compete in?}
\begin{Qlist}
	\item Soccer Robot Leagues
	\item Soccer Humanoid Leagues
	\item Soccer Simulation Leagues
	\item Rescue Robot League
	\item Rescue Simulation League
	\item @Home League
	\item @Work League
	\item Logistics League
\end{Qlist}}

\Qitem{ \Qq{What were the three main reasons for your decision?\newline} \Qline{3cm},\Qline{3cm},\Qline{3cm}.}

\Qitem{ \Qq{If you did not choose the Logistics League:\newline Was there a specific reason that you did not choose the Logistics League?} \hfill  \Qline{3cm} \newline \vskip 1pt \Qline{16cm}}

\section*{Teams already participating in RoboCup}

\Qitem{ \Qq{Is it easy for you to follow a RCLL game?} \hfill \QO{} Yes\hskip1.25cm \QO{} No}

\Qitem{ \Qq{How do you feel about the commentators?} \hfill confusing \Qrating{5} helpful}

\Qitem[a]{ \Qq{Have you talked to RCLL teams before?} \hfill \QO{} Yes\hskip1.25cm \QO{} No}

\Qitem[b]{ \Qq{If you answered yes:\newline Did that help you understand the game better?} \hfill  \QO{} Yes\hskip1.25cm \QO{} No}

\Qitem[c]{ \Qq{If you answered no:\newline Were the commentators and status boards sufficient information?} \hfill \QO{} Yes\hskip1.25cm \QO{} No}

\Qitem[a]{ \Qq{Have you thought about competing in the RCLL?} \hfill \QO{} Yes\hskip1.25cm \QO{} No}

\Qitem[b]{ \Qq{If you answered yes:\newline What do you feel is the main reason you have not joined yet?} \hfill \Qline{3cm}  \newline \vskip 1pt \Qline{16cm}}\vspace{0.3cm}

\Qitem[c]{ \Qq{If you answered no:\newline What do you feel is missing to make it more interesting for you?} \hfill \Qline{3cm} \newline \vskip 1pt \Qline{16cm}}\vspace{0.3cm}

\Qitem{ \Qq{The RCLL currently does not have a simulation league\newline Do you think having a simulation league would help joining the league?} \hfill \QO{} Yes\hskip1.25cm \QO{} No}

\Qitem{ \Qq{Describe your overall impression of the RCLL in three words:\newline} \Qline{3cm},\Qline{3cm},\Qline{3cm}.}

\subsection*{Teams that stopped competing or are considering to stop}

\Qitem[a]{ \Qq{Is there a specific reason why you stopped competing \newline or are considering it?} \hfill \QO{} Yes\hskip1.25cm \QO{} No}

\Qitem[b]{ \Qq{If you answered yes:\newline What do you feel is the main reason you no longer compete?} \hfill \Qline{3cm} \newline \vskip 1pt \Qline{16cm}}\vspace{0.3cm}

\Qitem[a]{ \Qq{Do you think about competing again?} \hfill \QO{} Yes\hskip1.25cm \QO{} No}

\Qitem[b]{ \Qq{If you answered yes:\newline What kind of help would most helpful to you?} \hfill \Qline{3cm} \newline \vskip 1pt \Qline{16cm}}\vspace{0.3cm}

\Qitem[c]{ \Qq{If you answered no:\newline What do you feel is deterring you the most?} \hfill \Qline{3cm} \newline \vskip 1pt \Qline{16cm}}\vspace{0.3cm}


%\Qitem{ \Qq{What do you think is the scientific motivation behind the RCLL?} \hfill \Qline{3cm} }
%
%\section*{Post-Game}
%
%\Qitem{ \Qq{Did you find the game interesting?} \hfill boring \Qrating{5} very interesting}
%
%\Qitem[a]{ \Qq{Now that you have seen the game,\newline did the scientific motiviation become clearer?} \hfill  \QO{} Yes\hskip1.25cm \QO{} No}
%\Qitem[b]{ \Qq{If you answered yes, what is it?} \hfill  \Qline{3cm}}
%
%\Qitem{ \Qq{Do you think having an influence on what happens during\newline the game helps you understand it?} \hfill \QO{} Yes\hskip1.25cm \QO{} No}
%
%\Qitem{ \Qq{Is having a direct influence on the game important to you?} \hfill \QO{} Yes\hskip1.25cm \QO{} No}
%
%\Qitem{ \Qq{Would you rather watch a game:}
%\begin{Qlist}
%	\item that you actively influence?
%	\item that you purely spectate?
%\end{Qlist}}
%\Qitem{ \Qq{Do you like having an option?}
%\begin{Qlist}
%\item Yes, this meets my need for autonomy.
%\item No, I'd rather have someone else decide for me.
%\item Not sure.
%\end{Qlist}
%}

%\Qitem{ \Qq{What kind of music do you like best?}
%\begin{Qlist}
%\item Johann Sebastian Bach
%\item Jazz
%\item The Beatles
%\item Other: \Qline{4cm}
%\end{Qlist}
%}

%\minisec{Please evaluate the following composers}
%\vskip.5em
%
%\Qitem[a]{ \Qq{Mozart} \Qtab{3cm}{horrible \Qrating{5} fantastic}}
%
%\Qitem[b]{ \Qq{Beethoven} \Qtab{3cm}{horrible \Qrating{5} fantastic}}
%
%\Qitem[c]{ \Qq{Johann S. Bach} \Qtab{3cm}{horrible \Qrating{5}
%fantastic}}
%
%\Qitem[d]{ \Qq{John Lennon}\Qtab{2.5cm}{\parbox[t]{3.3cm}{\raggedleft
%Uh, well, I do not like his music very much}
%\Qrating{7} \parbox[t]{3cm}{Oh, yes, you know, really great
%stuff}}}
%
%\Qitem[e]{ \Qq{Elvis}\Qtab{2.5cm}{\parbox[t]{3.3cm}{\raggedleft Uh,
%well, I do not like his music very much}
%\Qrating{7} \parbox[t]{3cm}{Oh, yes, you know, really great
%stuff}}}
%
%\section*{About this questionnaire}
%
%\Qitem{\Qq{Do you like the style so far?} \Qtab{5.5cm}{\QO{} Yes
%\hskip0.5cm \QO{} No}}
%
%\Qitem{\Qq{Is it really worth the ink?} \Qtab{5.5cm}{\QO{} Guess so.
%\hskip0.5cm \QO{} Probably not. \hskip0.5cm \QO{} Don't know.}}
%
%\Qitem{\Qq{How does this item look different from the previous one?}
%\Qtab{10.5cm}{\QO{}\Qtab{0.6cm}{Oh. Does it?}}\par
%\Qtab{10.5cm}{\QO{}\Qtab{0.6cm}{No clue. I just can't figure it out.
%So sorry.}}\par
%\Qtab{10.5cm}{\QO{}\Qtab{0.6cm}{I guess what you mean is that here
%the different answer options are below each other.}}\par
%}
%
%\Qitem[a]{ \Qq{Please describe your first impression.} \Qlines{2} }
%
%\Qitem[b]{ \Qq{In case you would like some more lines to write, here
%they are:} \Qlines{4} }
%
%\section*{Another feature}
%
%\noindent Now, here is another nice feature: you can automatically
%alter the background color of the items. Here is how it can look like.
%Because we are too lazy to think of new questions we will just ask you
%the same questions again. There is one difference to the above though,
%can you find it?
%
%\renewcommand{\QO}{$\ocircle$}% Use circles now instead of boxes.
%
%\section*{About you}
%
%\QItem{ \Qq{Your name}: \Qline{8cm} }
%
%\QItem{ \Qq{How old are you?} I am \Qline{1.5cm} years old.}
%
%\QItem{ \Qq{Are you in a good mood right now?} \hskip0.4cm \QO{}
%absolutely \hskip0.5cm \QO{} not really because: \Qline{3cm} }
%
%\QItem{ \Qq{Do you like having an option?}
%\begin{Qlist}
%\item Yes, this meets my need for autonomy.
%\item No, I'd rather have someone else decide for me.
%\item Not sure.
%\end{Qlist}
%}
%
%\QItem{ \Qq{What kind of music do you like best?}
%\begin{Qlist}
%\item Johann Sebastian Bach
%\item Jazz
%\item The Beatles
%\item Other: \Qline{4cm}
%\end{Qlist}
%}
%
%\minisec{Please evaluate the following composers}
%\vskip.5em
%
%\QItem[a]{ \Qq{Mozart} \Qtab{3cm}{horrible \Qrating{5} fantastic}}
%
%\QItem[b]{ \Qq{Beethoven} \Qtab{3cm}{horrible \Qrating{5} fantastic}}
%
%\QItem[c]{ \Qq{Johann S. Bach} \Qtab{3cm}{horrible \Qrating{5} fantastic}}
%
%\QItem[d]{ \Qq{John Lennon}\Qtab{2.5cm}{\parbox[t]{3.3cm}{\raggedleft
%Uh, well, I do not like his music very much}
%\Qrating{7} \parbox[t]{3cm}{Oh, yes, you know, really great
%stuff}}}
%
%\QItem[e]{ \Qq{Elvis}\Qtab{2.5cm}{\parbox[t]{3.3cm}{\raggedleft
%Uh, well, I do not like his music very much}
%\Qrating{7} \parbox[t]{3cm}{Oh, yes, you know, really great
%stuff}}}
%
%\section*{About this questionnaire}
%
%\QItem{\Qq{Do you like the style so far?} \Qtab{5.5cm}{\QO{} Yes
%\hskip0.5cm \QO{} No}}
%
%\QItem{\Qq{Is it really worth the ink?} \Qtab{5.5cm}{\QO{} Guess so.
%\hskip0.5cm \QO{} Probably not. \hskip0.5cm \QO{} Don't know.}}
%
%\QItem{\Qq{How does this item look different from the previous one?}
%\Qtab{10.5cm}{\QO{}\Qtab{0.6cm}{Oh. Does it?}}\par
%\Qtab{10.5cm}{\QO{}\Qtab{0.6cm}{No clue. I just can't figure it out.
%So sorry.}}\par
%\Qtab{10.5cm}{\QO{}\Qtab{0.6cm}{I guess what you mean is that here
%the different answer options are below each other.}}\par
%}
%
%\QItem[a]{ \Qq{Please describe your first impression.} \Qlines{2} }
%
%\QItem[b]{ \Qq{In case you would like some more lines to write, here
%they are:} \Qlines{4} }
%
\end{document}


Kontakt
Niklas Sebastian Eltester
eltester@fh-aachen.de

Kontakt / Impressum
